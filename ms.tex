\documentclass[12pt]{article}
\usepackage[]{graphicx}
\usepackage[]{color}
%% use \xspace to allow for space after a macro as necessary
\usepackage{xspace}
\usepackage[top=1in, bottom=1in, left=1.25in, right=1.25in]{geometry}
\usepackage{pdfpages}
%% maxwidth is the original width if it is less than linewidth
%% otherwise use linewidth (to make sure the graphics do not exceed the margin)
\makeatletter
\def\maxwidth{ %
  \ifdim\Gin@nat@width>\linewidth
    \linewidth
  \else
    \Gin@nat@width
  \fi
}
\makeatother

\definecolor{commentcol}{rgb}{0.345, 0.345, 0.945}
\newcommand{\comment}[1]{\textcolor{commentcol}{[[#1]]}}%

% general-purpose mathrm macros
\newcommand{\symsub}[2]{\ensuremath{#1_{\tiny \mathrm{#2}}}\xspace}
\newcommand{\mrm}[1]{\ensuremath{\mathrm{#1}}
\xspace}

\usepackage{framed}
\makeatletter
\newenvironment{kframe}{%
 \def\at@end@of@kframe{}%
 \ifinner\ifhmode%
  \def\at@end@of@kframe{\end{minipage}}%
  \begin{minipage}{\columnwidth}%
 \fi\fi%
 \def\FrameCommand##1{\hskip\@totalleftmargin \hskip-\fboxsep
 \colorbox{shadecolor}{##1}\hskip-\fboxsep
     % There is no \\@totalrightmargin, so:
     \hskip-\linewidth \hskip-\@totalleftmargin \hskip\columnwidth}%
 \MakeFramed {\advance\hsize-\width
   \@totalleftmargin\z@ \linewidth\hsize
   \@setminipage}}%
 {\par\unskip\endMakeFramed%
 \at@end@of@kframe}
\makeatother

\definecolor{randomcolor}{rgb}{.97, .27, .67}
\definecolor{shadecolor}{rgb}{.97, .97, .97}
\definecolor{messagecolor}{rgb}{0, 0, 0}
\definecolor{warningcolor}{rgb}{1, 0, 1}
\definecolor{errorcolor}{rgb}{1, 0, 0}
\newenvironment{knitrout}{}{} % an empty environment to be redefined in TeX

\usepackage{alltt}
\usepackage[sort]{natbib}
\usepackage{amsmath}
\usepackage{alltt}
\usepackage{hyperref}
\usepackage[utf8]{inputenc} % for accented characters
%% stuff for editing
%\usepackage[markup=nocolor,addedmarkup=bf,deletedmarkup=sout]{changes}
%% to suppress notes & comments: \usepackage[final]{changes}
\usepackage[backgroundcolor=lightgray]{todonotes}
\usepackage{setspace}
\bibliographystyle{chicago}
\title{Fitting phylogenetic generalized linear mixed-effect models using lme4 }
\author{Michael Li and Ben Bolker}
\date{\today}

\providecommand{\keywords}[1]{\textbf{\textit{Keywords:}} #1}
\IfFileExists{upquote.sty}{\usepackage{upquote}}{}
\begin{document}
\newcommand{\dbic}{\ensuremath \Delta \textrm{BIC}}

%% don't typeset BMB comments
\newcommand{\bmbhide}[1]{}
\newcommand{\bmb}[1]{{\color{blue} BB: #1}}

\newcommand{\fref}[1]{Figure~\ref{fig:#1}}

\newcommand{\ml}[1]{{\color{red} ML: #1}}

\newcommand{\add}[1]{{\color{blue} ADD: #1}}

%\SweaveOpts{concordance=TRUE}
%\SweaveOpts{concordance=TRUE}
\maketitle

\doublespacing

\keywords{phyloglmm ... }

\section{Introduction}
\ml{I will fill in appropriate citation later}


Given a phylogenetic tree, phylogenetic models aim to find parsimonious methods that can be used to link ecological phenomena with evolutionary processes that generate species and exhibit in their traits.
Ecologist have long considered the effects of species functional traits and environmental conditions on their phenotypic traits and also incorportating historical relationships of species.
Decades of work have created various different methods that enables researchers to analyze species relativeness problems/systems for different types of data and questions.
%For example, regression analyses of body temperature as a function of body size for animal within a clade, species compositions in different sites, etc. 
\ml{more examples...and reference}

The most widely used statistical method to model/fit related species data that accounts for phylogenetic structure is the phylogenetic regression (Felsenstein 1985). 
More recently, people use linear mixed model and generalized linear mixed model framework to model complex systems with phylogenetic structures. 


Despite the choice of modeling strageties, there isn't a convient platform to do all and limited in model complexities additional complexities in the systems. 

In this paper, we model it using lme4 and test it with simulated cases across platforms and fit it to dun (maybe more examples if we can find it) example.  

We also compare different platforms. 

\section{Methods}

We generate test data using a simple glmm framework that varies in complexity in random effect. 

\section{Phylogenetic Tree Transformation}
The standard problem of phylogenetic comparative methods is to analyze relationships among data where the observations are gathered from nodes (usually tips) of a phylogenetic tree. 

An alternative approach is to model the phylogenetic correlation as a gaussian process. 
In particular, suppose that the evolutionary process is a Brownian motion. 
In that case, the phylogenetic variability of a particular observation can be written as the sum of the evolutionary changes that occurred on all of the branches in the phylogeny in its past. 
If we set up the Z matrix appropriately, we can model everythign with a sequence of independent errors, rather than having to do a fancy things to impose a correlation structure on the random effects. 


\subsection{Platforms}

...


\section{Results}

\section{Discussion}

\section{Conclusion}

\end{document}

